Currently, we support and maintain R\-A\-P\-P-\/\-A\-P\-I(s) for the following programming languages\-:


\begin{DoxyItemize}
\item Python
\item Java\-Script
\item C++
\end{DoxyItemize}

Under the \href{https://github.com/rapp-project/rapp-api}{\tt rapp-\/api repository} you can find more information regarding implemented and tested Rapp-\/\-Platform A\-P\-I calls.

As the integration tests are written in Python and uses the \href{https://github.com/rapp-project/rapp-api/tree/master/python}{\tt Python-\/\-Rapp-\/\-A\-Pi} it is a good practice to start on the python-\/rapp-\/api.





\subsubsection*{Python R\-A\-P\-P-\/\-Platform-\/\-A\-P\-I Overview -\/ Usage}

A\-P\-I users are able to select from 2 A\-P\-I implementations\-:


\begin{DoxyItemize}
\item {\bfseries High level A\-P\-I}
\item {\bfseries Advanced A\-P\-I}
\end{DoxyItemize}

The first one allow A\-P\-I users to easily call R\-A\-P\-P P\-Latform Services through simple function calls. The second one is for advanced usage, delivered for expert developers. This is an object-\/oriented implementation. As we will later describe, the advanced A\-P\-I usage allow creation of Cloud Messages. Both Platform requests and responses are described by static objects.

{\bfseries Note}\-: The High Level A\-P\-I actually wraps the Advanced A\-P\-I.

\paragraph*{Advanced A\-P\-I usage}

{\ttfamily Rapp\-Platform\-Service} is the R\-A\-P\-P term for an established connection to the R\-A\-P\-P-\/\-Platform Services, over the www (World-\/\-Wide-\/\-Web). Each Platform Service has it's own unique Response and Request message.

The {\ttfamily Rapp\-Platform\-Service} class is used to establish connections to the R\-A\-P\-P-\/\-Platform Web-\/\-Services, while {\ttfamily Cloud\-Msg} objects include\-:
\begin{DoxyItemize}
\item {\ttfamily Request} object. R\-A\-P\-P-\/\-Platform Service specific Request message
\item {\ttfamily Response} object. R\-A\-P\-P-\/\-Platform Service specific Response message
\end{DoxyItemize}

```python from Rapp\-Cloud import Rapp\-Platform\-Service

svc\-Client = Rapp\-Platform\-Service(persistent=True, timeout=30000) ```

By Default it connects to the localhost, assuming that the R\-A\-P\-P Platform has been setup on the local machine. The constructor of the {\ttfamily Rapp\-Platform\-Service} class allow to specify the R\-A\-P\-P Platform parameters to connect to.

```python from Rapp\-Cloud import Rapp\-Platform\-Service

svc\-Client = Rapp\-Platform\-Service(address='R\-A\-P\-P\-\_\-\-P\-L\-A\-T\-F\-O\-R\-M\-\_\-\-I\-Pv4\-\_\-\-A\-D\-D\-R\-E\-S\-S', port='R\-A\-P\-P\-\_\-\-P\-L\-A\-T\-F\-O\-R\-M\-\_\-\-P\-O\-R\-T\-\_\-\-N\-U\-M\-B\-E\-R', protocol='http') ```

{\ttfamily Rapp\-Platform\-Service} object constructor allow to set\-:


\begin{DoxyItemize}
\item persistent (Boolean)\-: Force peristent connections. {\bfseries Defaults to True}
\item timeout (Integer)\-: Client timeout value. This is the timeout value waiting for a response from the R\-A\-P\-P Platform. {\bfseries Defaults to infinity}
\item address (String)\-: The R\-A\-P\-P Platform I\-Pv4 address to connect to. {\bfseries Defaults to 'localhost'}
\item port (String)\-: The R\-A\-P\-P Platform listening port. {\bfseries Defaults to \char`\"{}9001\char`\"{}}
\item protocol (String)\-: The configured application protocol for the R\-A\-P\-P Platform. Valid values are \char`\"{}$\ast$$\ast$https$\ast$$\ast$\char`\"{} and \char`\"{}$\ast$$\ast$http$\ast$$\ast$\char`\"{}. {\bfseries Defaults to \char`\"{}http\char`\"{}}
\end{DoxyItemize}

The {\ttfamily persistent} and {\ttfamily timeout} properties of a {\ttfamily Rapp\-Platform\-Service} object are public members and can be set using the {\bfseries dot} (.) notation\-:

```py svc\-Client = Rapp\-Platform\-Service() svc\-Client.\-persistent = True svc\-Client.\-timeout = 30000 ```

{\ttfamily Cloud\-Msg} objects are feed to the {\ttfamily Rapp\-Platform\-Service} object to specific R\-A\-P\-P-\/\-Platform Services. {\ttfamily Cloud\-Msg} classes can be imported from the Cloud\-Msgs submodule of the Rapp\-Cloud module\-:

```py from Rapp\-Cloud.\-Cloud\-Msgs import Face\-Detection ```

The above line of code is used as an example of importing the {\ttfamily Face\-Detection} Cloud\-Msg class.

A complete description on available Cloud\-Msg classes as long as their Request and Response message classes is available here

Cloud\-Msg objects hold a Request and a Response object\-:

```py from Rapp\-Cloud.\-Cloud\-Msgs import Face\-Detection face\-Detect\-Msg = Face\-Detection()

req\-Obj = face\-Detect\-Msg.\-req resp\-Obj = face\-Detect\-Msg.\-resp ```

Request and Response objects of a Cloud\-Msg can be serialized to a dictionary\-:

```py req\-Dict = face\-Detect\-Msg.\-req.\-serialize() print req\-Dict \begin{quotation}
\begin{quotation}
\{fast\-: False, image\-File\-Path\-: ''\}

\end{quotation}


\end{quotation}


resp\-Dict = face\-Detect\-Msg.\-resp.\-serialize() print resp\-Dict \begin{quotation}
\begin{quotation}
\{faces\-: \mbox{[}\mbox{]}, error\-: ''\}

\end{quotation}


\end{quotation}
```

Cloud\-Msg Request property values can be set through the {\ttfamily req} property of the Cloud\-Msg object. or as keyword arguments to the constructor of a Cloud\-Msg\-:

```py from Rapp\-Cloud.\-Cloud\-Msgs import Face\-Detection

msg = Face\-Detection(image\-Filepath='/tmp/face-\/sample.png') print msg.\-req.\-serialize() \begin{quotation}
\begin{quotation}
\{fast\-: False, image\-Filepath\-: '/tmp/face-\/sample.png'\}

\end{quotation}


\end{quotation}


msg.\-req.\-fast = True print msg.\-req.\-serialize() \begin{quotation}
\begin{quotation}
\{fast\-: True, image\-Filepath\-: '/tmp/face-\/sample.png'\}

\end{quotation}


\end{quotation}
```

{\ttfamily Rapp\-Platfomr\-Service} objects have a {\ttfamily .call()} method for calling R\-A\-P\-P-\/\-Platform Services\-:

```py class Rapp\-Platform\-Service\-: ...

def call(self, msg=None)\-: ... return self.\-resp

... ```

The {\ttfamily .call()} method returns the Response object.

```py svc\-Client= Rapp\-Platform\-Service() msg = Face\-Detection() msg.\-req.\-fast = True msg.\-req.\-image\-Filepath = '/tmp/face-\/sample.png'

response = svc\-Client.\-call(msg) print response.\-faces print response.\-error

```

Cloud\-Msg objects are passed as argument to the {\ttfamily .call()} method of the {\ttfamily Rapp\-Platform\-Service} object\-:

```py svc\-Client= Rapp\-Platform\-Service() msg = Face\-Detection(image\-Filepath='/tmp/face-\/sample.png') response = svc\-Client.\-call(msg) ```

{\ttfamily Cloud\-Msg} objects can also be passed to the constructor of the {\ttfamily Rapp\-Platform\-Service} class\-:

```py face\-Msg = Face\-Detection(image\-Filepath='/tmp/face-\/sample.png') svc\-Client= Rapp\-Platform\-Service(msg=face\-Msg, timeout=15000) response = svc\-Client.\-call() ```

{\bfseries Note}\-: Calling several different R\-A\-P\-P-\/\-Platform Services is done by passing the service specific Cloud Message objects to the {\ttfamily .call()} of the {\ttfamily Rapp\-Platform\-Service} object.

The following example creates a {\ttfamily Face\-Detection} and a {\ttfamily Qr\-Detection} Cloud\-Msg to call both the Face-\/\-Detection and Qr-\/\-Detection R\-A\-P\-P-\/\-Platform Services.

```py from Rapp\-Cloud import Rapp\-Platform\-Service from Rapp\-Cloud.\-Cloud\-Msgs import ( Face\-Detection, Qr\-Detection)

svc\-Client = Rapp\-Platform\-Service(timeout=1000) face\-Msg = Face\-Detection(fast=True, image\-Filepath='/tmp/face-\/sample.png') qr\-Msg = Qr\-Detection() qr\-Msg.\-req.\-image\-Filepath = '/tmp/qr-\/sample.png'

fd\-Resp = svc\-Client.\-call(face\-Msg) print \char`\"{}\-Found \%s Faces\char`\"{} len(fd\-Resp.\-faces)

qr\-Resp = svc\-Client.\-call(qr\-Msg) print \char`\"{}\-Found \%s Q\-Rs\-: \%s\char`\"{} \%(len(qr\-Resp.\-qr\-\_\-centers), qr\-Resp.\-qr\-\_\-messages)

```

\paragraph*{High Level A\-P\-I usage}

Like previously mentioned, A\-P\-I users can also use the High Level implementation of the R\-A\-P\-P Platform A\-P\-I. Benefits from using this implementation is lack of knowledge of how Cloud Messages and Rapp\-Platform\-Service are used. Calls to the R\-A\-P\-P Platform are done through simple function calls, under the Rapp\-Platform\-A\-P\-I module.

Below is an example of performing a query to the ontologyi, hosted on the R\-A\-P\-P Platform, using the High Level A\-P\-I implementation\-:

```python from Rapp\-Cloud import Rapp\-Platform\-A\-P\-I ch = Rapp\-Platform\-A\-P\-I()

response = ch.\-ontology\-Subclasses(\char`\"{}\-Oven\char`\"{})

print response \begin{quotation}
\begin{quotation}
\{'results'\-: \mbox{[}u'\href{http://knowrob.org/kb/knowrob.owl#MicrowaveOven',}{\tt http\-://knowrob.\-org/kb/knowrob.\-owl\#\-Microwave\-Oven',} u'\href{http://knowrob.org/kb/knowrob.owl#RegularOven',}{\tt http\-://knowrob.\-org/kb/knowrob.\-owl\#\-Regular\-Oven',} u'\href{http://knowrob.org/kb/knowrob.owl#ToasterOven'}{\tt http\-://knowrob.\-org/kb/knowrob.\-owl\#\-Toaster\-Oven'}\mbox{]}, 'error'\-: u''\}

\end{quotation}


\end{quotation}
```

The Rapp\-Platform\-A\-P\-I usage and calls are fully documented \href{https://github.com/rapp-project/rapp-api/tree/master/python/RappCloud}{\tt here}, also with examples of usage.





\subsubsection*{Example -\/ Implementing the Face\-Detection A\-P\-I call.}

Lets say we want to implement the {\ttfamily Face\-Detection} Cloud Message.

The face detection R\-A\-P\-P Platform Web Service has a Request and Response object

{\bfseries Web-\/\-Service Request}


\begin{DoxyItemize}
\item {\ttfamily fast} (Boolean)\-: If true, detection will take less time but it will be less accurate
\item {\ttfamily file} (File)\-: Image file.
\end{DoxyItemize}

{\bfseries Web-\/\-Service Response}


\begin{DoxyItemize}
\item {\ttfamily faces} (Array)\-: An array of the detected faces coordinates (point2\-D), on the image frame.
\item {\ttfamily error} (String)\-: Error message.
\end{DoxyItemize}

Start by creating the python source file for the Face\-Deteciton Cloud Message implementation. Head to the \href{https://github.com/rapp-project/rapp-api/tree/master/python/RappCloud/CloudMsgs}{\tt Rapp\-Cloud/\-Cloud\-Msgs} directory of the Rapp\-Cloud module and create a file named {\bfseries Face\-Detection.\-py}

Cloud Messages classes inherit from the {\ttfamily Cloud\-Msg} class and {\ttfamily Request} and {\ttfamily Response} classes inherit from {\ttfamily Cloud\-Request} and {\ttfamily Cloud\-Response} classes respectively. So first import those classes and write the structure of the {\ttfamily Face\-Detection} Cloud Message\-:

```python from Cloud import ( Cloud\-Msg, Cloud\-Request, Cloud\-Response)

class Face\-Detection(\-Cloud\-Msg)\-: \begin{DoxyVerb}class Request(CloudRequest):
    def __init__(self, **kwargs):
        pass


class Response(CloudResponse):
    def __init__(self, **kwargs):
        pass


def __init__(self, **kwargs):
    pass
\end{DoxyVerb}
 ```

Add the appropriate properties to the Request and Response classes. Property names can differ from the Web-\/\-Service request and response property names. Mapping from implemented property names to actual request payload will be studied later on\-:

```python class Request(\-Cloud\-Request)\-: \begin{DoxyVerb}def __init__(self, **kwargs):
    ## File path to the image file
    self.imageFilepath = ''
    ## If true, detection will take less time but it will be less accurate
    self.fast = False
    ## Apply keyword arguments to the Request object.
    super(FaceDetection.Request, self).__init__(**kwargs)
\end{DoxyVerb}


class Response(\-Cloud\-Response)\-: \begin{DoxyVerb}def __init__(self, **kwargs):
    ## Error message
    self.error = ''
    ## Detected faces. Array of face objects.
    self.faces = []
    ## Apply keyword arguments to the Request object.
    super(FaceDetection.Response, self).__init__(**kwargs)
\end{DoxyVerb}
 ```

{\bfseries Notice calling {\ttfamily Cloud\-Response} and {\ttfamily Cloud\-Request} construcors.}

Remember that a {\ttfamily Request} class must implement two member methods, {\ttfamily make\-\_\-payload()} and {\ttfamily make\-\_\-files}. For the payload it's the {\ttfamily fast} property and image\-Filepath is a file.

```python from Rapp\-Cloud.\-Objects import ( Payload, File)

class Request(\-Cloud\-Request)\-: \begin{DoxyVerb}def __init__(self, **kwargs):
    ## File path to the image file
    self.imageFilepath = ''
    ## If true, detection will take less time but it will be less accurate
    self.fast = False
    ## Apply keyword arguments to the Request object.
    super(FaceDetection.Request, self).__init__(**kwargs)

def make_payload(self):
    return Payload(fast=self.fast)

def make_files(self):
    return [File(filepath=self.path, postfield="file")]
\end{DoxyVerb}
 ```

Next, you have to instantiate a {\ttfamily Request} and {\ttfamily Response} objects for the {\ttfamily Face\-Detection} class to hold\-:

```python class Face\-Detection(\-Cloud\-Msg)\-: ...

def {\bfseries init}(self, $\ast$$\ast$kwargs)\-: \section*{Create and hold the Request object for this Cloud\-Msg}

self.\-req = Face\-Detection.\-Request() \section*{Create and hold the Response object for this Cloud\-Msg}

self.\-resp = Face\-Detection.\-Response() ```

Each R\-A\-P\-P Platform Web Service has a unique service name resolving to a url name/path. The service name for the Face Detection R\-A\-P\-P Platform Web Service is\-:

```shell face\-\_\-detection ```

{\ttfamily Cloud\-Msg} constructor takes as input the service name\-:

```python class Face\-Detection(\-Cloud\-Msg)\-: ...

def {\bfseries init}(self, $\ast$$\ast$kwargs)\-: \section*{Create and hold the Request object for this Cloud\-Msg}

self.\-req = Face\-Detection.\-Request() \section*{Create and hold the Response object for this Cloud\-Msg}

self.\-resp = Face\-Detection.\-Response() super(\-Face\-Detection, self).\-\_\-\-\_\-init\-\_\-\-\_\-(svcname='face\-\_\-detection', $\ast$$\ast$kwargs) ```

{\bfseries Note}\-: Dont forget to document the code using {\bfseries doxygen}

Below is the complete Face\-Detection.\-py file

```python from Rapp\-Cloud.\-Objects import ( File, Payload)

from Cloud import ( Cloud\-Msg, Cloud\-Request, Cloud\-Response)

class Face\-Detection(\-Cloud\-Msg)\-: \char`\"{}\char`\"{}\char`\"{} Face Detection Cloud\-Msg object\char`\"{}\char`\"{}\char`\"{}

class Request(\-Cloud\-Request)\-: \char`\"{}\char`\"{}\char`\"{} Face Detection Cloud Request object. Face\-Detection.\-Request \char`\"{}\char`\"{}\char`\"{} def {\bfseries init}(self, $\ast$$\ast$kwargs)\-: \char`\"{}\char`\"{}"! Constructor 
\begin{DoxyParams}{Parameters}
{\em $\ast$$\ast$kwargs} & -\/ Keyword arguments. Apply values to the request attributes.
\begin{DoxyItemize}
\item image\-Filepath
\item fast"
\end{DoxyItemize}\\
\hline
\end{DoxyParams}
\subsection*{File path to the image to load. This is the image to perform}

\section*{face-\/detection on.}

self.\-image\-Filepath = '' \subsection*{If true, detection will take less time but it will be less}

\section*{accurate}

self.\-fast = False \section*{Apply passed keyword arguments to the Request object.}

super(Face\-Detection.\-Request, self).\-\_\-\-\_\-init\-\_\-\-\_\-($\ast$$\ast$kwargs)

\begin{DoxyVerb}    def make_payload(self):
        """ Create and return the Payload of the Request. """
        return Payload(fast=self.fast)

    def make_files(self):
        """ Create and return Array of File objects of the Request. """
        return [File(self.imageFilepath, postfield='file')]

    class Response(CloudResponse):
        """ Face Detection Cloud Response object. FaceDetection.Response """
        def __init__(self, **kwargs):
            """!
            Constructor

            @param **kwargs - Keyword arguments. Apply values to the request attributes.
            - @ref error
            - @ref faces
            """
            ## Error message
            self.error = ''
            ## Detected faces. Array of face objects. TODO create face object.
            self.faces = []
            ## Apply passed keyword arguments to the Request object.
            super(FaceDetection.Response, self).__init__(**kwargs)


    def __init__(self, **kwargs):
        """!
        Constructor

        @param **kwargs - Keyword arguments. Apply values to the request attributes.
            - @ref Request.fast
            - @ref Request.imageFilepath
        """

        # Create and hold the Request object for this CloudMsg
        self.req = FaceDetection.Request()
        # Create and hold the Response object for this CloudMsg
        self.resp = FaceDetection.Response()
        super(FaceDetection, self).__init__(svcname='face_detection', **kwargs)
\end{DoxyVerb}
 ```

Finally append the following line of code in the {\ttfamily Rapp\-Cloud/\-Cloud\-Msgs/\-\_\-\-\_\-init\-\_\-\-\_\-.\-py} file\-:

```python from Face\-Detection import Face\-Detection ```

Now everything is in place to call the newly created Face detection R\-A\-P\-P Platform Service, using the python implementation of the rapp-\/platform-\/api. An example is presented below\-:

```python from Rapp\-Cloud.\-Cloud\-Msgs import Face\-Detection from Rapp\-Cloud import Rapp\-Platform\-Service

svc\-Client = Rapp\-Platform\-Service(persistent=True, timeout=30000) msg = Face\-Detection(image\-Filepath=\char`\"{}\-P\-A\-T\-H\char`\"{}, fast=True)

response svc\-Client.\-call(msg)

if response.\-error\-: print \char`\"{}\-An error has occured\-: \%s\char`\"{} response.\-error else\-: print response.\-faces ```

If you want to also include it in the High Level A\-P\-I implementation, you will have to modify the \href{https://github.com/rapp-project/rapp-api/blob/master/python/RappCloud/RappPlatformApi.py}{\tt Rapp\-Platform\-Api.\-py} file.

First import the {\ttfamily Face\-Detection} Cloud Message, that was previously implemented\-:

```py ...

from Cloud\-Msgs import Face\-Detection ```

Next, we must implemend the {\ttfamily face\-Detection} method for the Rapp\-Platform\-A\-P\-I class. Input arguments to the method are\-:


\begin{DoxyItemize}
\item image\-Filepath\-: Path to the image file to perform face detection on.
\item fast\-: Force fast detection. If true, detection takes less time but it will be less accurate
\end{DoxyItemize}

The output is a python {\ttfamily dict} of the response fields\-:


\begin{DoxyItemize}
\item faces\-: Detected faces
\item error\-: Error message, if one occures.
\end{DoxyItemize}

The {\ttfamily face\-Detection} method implementation must be as presented below

```py ...

class Rapp\-Platform\-A\-P\-I()\-: \char`\"{}\char`\"{}\char`\"{} R\-A\-P\-P Platform simple A\-P\-I implementation \char`\"{}\char`\"{}\char`\"{} def {\bfseries init}(self)\-: self.\-svc\-\_\-caller = Rapp\-Platform\-Service()

...

def face\-Detection(self, image\-Filepath, fast = False)\-: \char`\"{}\char`\"{}"! Face detection A\-P\-I service call.  image\-Filepath\-: string 
\begin{DoxyParams}{Parameters}
{\em image\-Filepath} & Path to the image file.  fast\-: bool \\
\hline
{\em fast} & Perform fast detection. If true, detection will take less time but it will be less accurate. \-: dict \\
\hline
\end{DoxyParams}
\begin{DoxyReturn}{Returns}
\-: Returns a dictionary of the service call response. \{'faces'\-: \mbox{[}\mbox{]}, 'error'\-: ''\} \char`\"{}\char`\"{}" msg = Face\-Detection() try\-: msg.\-req.\-image\-Filepath = image\-Filepath response = self.\-svc\-\_\-caller.\-call(msg) except Exception as e\-: response = Face\-Detection.\-Response(error=str(e)) return \{ 'faces'\-: response.\-faces, 'error'\-: response.\-error \} ```
\end{DoxyReturn}
{\bfseries Note}\-: Make sure to launch both the back-\/end and the, listening for requests, H\-O\-P Web Server, of the R\-A\-P\-P Platform, before executing the above example. \href{https://github.com/rapp-project/rapp-platform/wiki/How-do-I-launch-the-RAPP-Platform%3F}{\tt Here} you can find instructions on how to launch the R\-A\-P\-P Platform. 