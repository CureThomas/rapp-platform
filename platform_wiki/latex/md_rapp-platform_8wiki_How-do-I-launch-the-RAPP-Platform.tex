This is quite easy! Just follow the instructions below which describe how to launch the deploy scripts. These scripts are located in the \href{https://github.com/rapp-project/rapp-platform-scripts}{\tt rapp-\/platform-\/scripts} repository in the folder {\ttfamily /deploy/}.

There are two files aimed for deployment\-:


\begin{DoxyItemize}
\item {\ttfamily deploy\-\_\-rapp\-\_\-ros.\-sh}\-: Deploys the R\-A\-P\-P Platform back-\/end, i.\-e. all the R\-O\-S nodes
\item {\ttfamily deploy\-\_\-web\-\_\-services.\-sh}\-: Deploys the corresponding H\-O\-P services
\end{DoxyItemize}

If you want to deploy the R\-A\-P\-P Platform in the background you can use {\ttfamily screen}. Just follow the next steps\-:


\begin{DoxyItemize}
\item {\ttfamily screen}
\item {\ttfamily ./deploy\-\_\-rapp\-\_\-ros.sh}
\item Press Ctrl + a + d to detach
\item {\ttfamily screen}
\item {\ttfamily ./deploy\-\_\-web\-\_\-services}
\item Press Ctrl + a + d to detach
\item {\ttfamily screen -\/ls} to check that 2 screen sessions exist
\end{DoxyItemize}

To reattach to screen session\-: ``` screen -\/r \mbox{[}pid.\mbox{]}tty.\-host ``` The screen step is for running rapp\-\_\-ros and web\-\_\-services on detached terminals which is useful, for example in the case where you want to connect via ssh to a remote computer, launch the processes and keep them running even after closing the connection. Alternatively, you can open two terminals and run one script on each, without including the screen commands. It is imperative for the terminals to remain open for the processes to remain active.

Screen how-\/to\-: \href{http://www.rackaid.com/blog/linux-screen-tutorial-and-how-to/}{\tt http\-://www.\-rackaid.\-com/blog/linux-\/screen-\/tutorial-\/and-\/how-\/to/} 