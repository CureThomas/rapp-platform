 \subsection*{$\ast$$\ast$\-List of contents$\ast$$\ast$ }


\begin{DoxyEnumerate}
\item \href{https://github.com/rapp-project/rapp-platform/wiki/#components}{\tt Components}
\item \href{https://github.com/rapp-project/rapp-platform/wiki/#package-launch}{\tt Package launch}
\item \href{https://github.com/rapp-project/rapp-platform/wiki/#prepare-new-map-manually}{\tt Prepare new map manually}
\item \href{https://github.com/rapp-project/rapp-platform/wiki/#ros-services}{\tt R\-O\-S services}
\item \href{https://github.com/rapp-project/rapp-platform/wiki/#web-services}{\tt W\-E\-B services}
\end{DoxyEnumerate}

\subsection*{Components}

\paragraph*{1. {\itshape rapp\-\_\-path\-\_\-planning}}

Rapp\-\_\-path\-\_\-planning is used in the R\-A\-P\-P case to plan path from given pose to given goal. User can costomize the path planning module with following parameters\-:
\begin{DoxyItemize}
\item pebuild map -\/ avaliable maps are stored \href{https://github.com/rapp-project/rapp-platform/tree/master/rapp_map_server/maps}{\tt here},
\item planning algorithm -\/ {\bfseries for now, only \href{https://en.wikipedia.org/wiki/Dijkstra%27s_algorithm}{\tt dijkstra} is avaliable},
\item robot type -\/ customizes costmap for planning module. {\bfseries For now only \href{https://www.aldebaran.com/en/humanoid-robot/nao-robot}{\tt N\-A\-O} is supported}.
\end{DoxyItemize}

\paragraph*{2. {\itshape rapp\-\_\-map\-\_\-server}}

Rapp\-\_\-map\-\_\-server delivers prebuild maps to rapp\-\_\-path\-\_\-planning component. All avaliable maps are contained \href{https://github.com/rapp-project/rapp-platform/tree/master/rapp_map_server/maps}{\tt here}.

This component is based on the R\-O\-S package\-: \href{http://wiki.ros.org/map_server}{\tt map\-\_\-server}. The rapp\-\_\-map\-\_\-server reads .png and .yaml files and publishes map as \href{http://docs.ros.org/jade/api/nav_msgs/html/msg/OccupancyGrid.html}{\tt Occupacy\-Grid} data. R\-A\-P\-P case needs run-\/time changing map publication, thus the rapp\-\_\-map\-\_\-server extands map\-\_\-server functionality. The rapp\-\_\-map\-\_\-server enables user run-\/time changes of map. It subscribes to R\-O\-S parameter\-: {\ttfamily rospy.\-set\-\_\-param(node\-Name+\char`\"{}/set\-Map\char`\"{}, map\-\_\-path)} and publishes the map specified in map\-\_\-path. Examplary map changing request is presented below. ```python nodename = rospy.\-get\-\_\-name() map\-\_\-path = \char`\"{}/home/rapp/rapp\-\_\-platform/rapp-\/platform-\/catkin-\/ws/src/rapp-\/platform/rapp\-\_\-map\-\_\-server/maps/empty.\-yaml\char`\"{} rospy.\-set\-\_\-param(nodename+/set\-Map, map\-\_\-path) {\ttfamily  A R\-O\-S service exists to store new maps in each user's workspace, called}upload\-\_\-map{\ttfamily . Then each application can invoke the}plan\-Path2\-D``` service, providing the map's name (among others) as input argument.

\subsection*{Package launch}

\paragraph*{$\ast$\-Standard launch$\ast$}

Launches the {\bfseries path planning} node and can be launched using ```bash roslaunch rapp\-\_\-path\-\_\-planning path\-\_\-planning.\-launch ```

\subsection*{Prepare new map manually}

New map files have to be compatible with the \href{http://wiki.ros.org/map_server}{\tt map\-\_\-server} package. Detailed description of files and their parameters can be found \href{http://wiki.ros.org/map_server/#Map_format}{\tt here}. Exemplary map files can be found \href{https://github.com/rapp-project/rapp-platform/tree/devel/rapp_path_planning/rapp_map_server/maps}{\tt here}.


\begin{DoxyEnumerate}
\item Prepare P\-N\-G file\-:\par
 a) Dimension desired area,\par
 b) Scale measurements by a desired factor. Chosen factor represents the map resolution! \par
 c) Draw obstacles and walls with black color carefully. Size of obstacles and walls is very important and every pixel makes difference!\par
 d) Rest of the map should be left white.\par
 e) Export your image to the .png file. Name of the file is important, so choose wisely!\par

\item Write configuration file\-:\par
 a) Create empty file\-: {\ttfamily $<$map\-\_\-name$>$.yaml}\par
 b) Open the file and set following configuration parameters\-:\par

\end{DoxyEnumerate}

``` \section*{Name of .png file.}

image\-: $<$png\-\_\-file\-\_\-name$>$ \# example -\/$>$ image\-: test\-\_\-map.\-png

\section*{The map's resolution \mbox{[}meters / pixel\mbox{]}}

resolution\-: $<$resolution$>$ \# example -\/$>$ resolution\-: 0.\-1

\section*{The map's origin. 2\-D pose of the map origin. \mbox{[}x, y, yaw\mbox{]}}

origin\-: $<$map\-\_\-origin$>$ \# example -\/$>$ origin\-: \mbox{[}0.\-0, 0.\-0, 0.\-0\mbox{]}

\section*{Whether the occupied / unoccupied pixels must be negated}

negate\-: $<$negate$>$ \# example -\/$>$ negate\-: 0

\section*{Pixels with occupancy probability greater than this threshold are considered completely occupied.}

occupied\-\_\-thresh\-: 

\# example -\/$>$ occupied\-\_\-thresh\-: 0.\-65

\section*{Pixels with occupancy probability less than this threshold are considered completely free.}

free\-\_\-thresh\-: 

\# example -\/$>$ free\-\_\-thresh\-: 0.\-196 ``` \subsection*{R\-O\-S Services}

\paragraph*{2\-D path planning}

Service U\-R\-L\-: {\ttfamily /rapp/rapp\-\_\-path\-\_\-planning/plan\-Path2\-D}

Service type\-: ```bash \section*{Contains name to the desired map}

string map\-\_\-name \section*{Contains type of the robot. It is required to determine it's parameters (footprint etc.)}

string robot\-\_\-type \section*{Contains path planning algorithm name}

string algorithm \section*{Contains start pose of the robot}

geometry\-\_\-msgs/\-Pose\-Stamped start \section*{Contains goal pose of the robot}

\subsection*{geometry\-\_\-msgs/\-Pose\-Stamped goal }

\section*{status of the service}

\section*{plan\-\_\-found\-:}

\section*{$\ast$ 0 \-: path cannot be planned.}

\section*{$\ast$ 1 \-: path found}

\section*{$\ast$ 2 \-: wrong map name}

\section*{$\ast$ 3 \-: wrong robot type}

\section*{$\ast$ 4 \-: wrong algorithm}

uint8 plan\-\_\-found \section*{error\-\_\-message \-: error explanation}

string error\-\_\-message \section*{path \-: vector of Pose\-Stamped objects}

\section*{if plan\-\_\-found is true, this is an array of waypoints from start to goal, where the first one equals start and the last one equals goal}

geometry\-\_\-msgs/\-Pose\-Stamped\mbox{[}\mbox{]} path ```

\subsubsection*{map file upload}

Service U\-R\-L\-: {\ttfamily /rapp/rapp\-\_\-path\-\_\-planning/upload\-\_\-map}

Service type\-: ```bash \section*{The end user's username, since the uploaded map is personal}

string user\-\_\-name \section*{The map's name. Must be unique for this user}

string map\-\_\-name \section*{The map's resolution}

float32 resolution \section*{R\-O\-S-\/specific\-: The map's origin}

float32\mbox{[}\mbox{]} origin \section*{R\-O\-S-\/specific\-: Whether the occupied / unoccupied pixels must be negated}

int16 negate \section*{Occupied threshold}

float32 occupied\-\_\-thresh \section*{Unoccupied threshold}

float32 free\-\_\-thresh \section*{File size for sanity checks}

uint32 file\-\_\-size \section*{The map data}

\subsection*{char\mbox{[}\mbox{]} data }

byte status ``` More information on the Occupancy Grid Map representation can be found \href{http://docs.ros.org/jade/api/nav_msgs/html/msg/OccupancyGrid.html}{\tt here}

\subsection*{Web services}

\subsubsection*{Path planning 2\-D}

\paragraph*{U\-R\-L}

{\ttfamily localhost\-:9001/hop/path\-\_\-planning\-\_\-path\-\_\-2d}

\paragraph*{Input / Output}

``` Input = \{ \char`\"{}map\-\_\-name\char`\"{}\-: “\-T\-H\-E\-\_\-\-P\-R\-E\-S\-T\-O\-R\-E\-D\-\_\-\-M\-A\-P\-\_\-\-N\-A\-M\-E”, \char`\"{}robot\-\_\-type\char`\"{}\-: \char`\"{}\-Nao\char`\"{}, \char`\"{}algorithm\char`\"{}\-: \char`\"{}dijkstra\char`\"{}, \char`\"{}start\char`\"{}\-: \{x\-: 0, y\-: 10\}, \char`\"{}goal\char`\"{}\-: \{x\-: 10, y\-: 0\} \} {\ttfamily  } Output = \{ \char`\"{}plan\-\_\-found\char`\"{}\-: 0, \char`\"{}path\char`\"{}\-: \mbox{[}\{x\-: 0, y\-: 10\}, \{x\-: ... \mbox{]}, \char`\"{}error\char`\"{}\-: \char`\"{}\char`\"{} \} ```

\subsubsection*{Upload map}

\paragraph*{U\-R\-L}

{\ttfamily localhost\-:9001/hop/path\-\_\-planning\-\_\-upload\-\_\-map}

\paragraph*{Input / Output}

``` Input = \{ \char`\"{}png\-\_\-file\char`\"{}\-: “map.\-png”, \char`\"{}yaml\-\_\-file\char`\"{}\-: \char`\"{}map.\-yaml\char`\"{}, \char`\"{}map\-\_\-name\char`\"{}\-: \char`\"{}simple\-\_\-map\-\_\-1\char`\"{} \} {\ttfamily  } Output = \{ \char`\"{}error\char`\"{}\-: \char`\"{}\char`\"{} \} ```

The full documentation exists \href{https://github.com/rapp-project/rapp-platform/tree/master/rapp_web_services/services#path-planning-plan-path-2d}{\tt here} and \href{https://github.com/rapp-project/rapp-platform/tree/master/rapp_web_services/services#path-planning-upload-map}{\tt here}.

\subparagraph*{Author}

\href{https://github.com/dudekw}{\tt Wojciech Dudek} 