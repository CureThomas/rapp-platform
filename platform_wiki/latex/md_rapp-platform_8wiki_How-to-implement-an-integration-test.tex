It is recommended to study on \href{https://github.com/rapp-project/rapp-platform/wiki/RAPP-Testing-Tools}{\tt Rapp-\/\-Testing-\/\-Tools} first.

Basic documentation on \char`\"{}$\ast$$\ast$\-Developing Integration Tests$\ast$$\ast$\char`\"{} can be found \href{https://github.com/rapp-project/rapp-platform/tree/master/rapp_testing_tools}{\tt here}.

A more-\/in-\/depth information on how to write your first integration test is presented here.

The \href{https://github.com/rapp-project/rapp-platform/blob/master/rapp_testing_tools/scripts/default_tests/template.py}{\tt template.\-py} will be used as a reference.

Each integration test must have the following characteristics\-:


\begin{DoxyItemize}
\item Each test is written as a seperate python source file (.py).
\item Each test is implemented as a Rapp\-Interface\-Test class.
\item The Rapp\-Interface\-Test class must define two member methods\-:
\begin{DoxyItemize}
\item execute()
\item validate()
\end{DoxyItemize}
\item The Python R\-A\-P\-P-\/\-A\-P\-I is imported into our code in order to invoke Platform. Web Service requests.
\end{DoxyItemize}

So lets first create our Rapp\-Interface\-Test class which includes the aforementioned member methods.

```python class Rapp\-Interface\-Test\-: \char`\"{}\char`\"{}\char`\"{} Integration test class \char`\"{}\char`\"{}\char`\"{} def {\bfseries init}(self)\-: pass

def execute(self)\-: pass

def validate(self)\-: pass ```

Next we need to include the Python R\-A\-P\-P-\/\-A\-P\-I module and create an in instance of the Rapp\-Cloud class which will allow us to invoke Platform Web Service calls.

```python from Rapp\-Cloud import Rapp\-Cloud

class Rapp\-Interface\-Test\-: \char`\"{}\char`\"{}\char`\"{} Integration test class \char`\"{}\char`\"{}\char`\"{} def {\bfseries init}(self)\-: self.\-rapp\-Cloud = Rapp\-Cloud() pass

```

Load any files required for integration test on initialization of the Rapp\-Interface\-Test instance and declare the validation values.

```python def {\bfseries init}(self)\-: self.\-rapp\-Cloud = Rapp\-Cloud()

\section*{Set the file path}

self.\-file\-\_\-uri = 'P\-A\-T\-H\-\_\-\-T\-O\-\_\-\-L\-O\-A\-D\-E\-D\-\_\-\-F\-I\-L\-E' \section*{Set the valid results.}

self.\-valid\-\_\-results = \{\} ```

Next we need to implement the validation function, {\bfseries validate(self)}.

```python def validate(self, response)\-: error = response\mbox{[}'error'\mbox{]} if error != \char`\"{}\char`\"{}\-: return \mbox{[}error, self.\-elapsed\-\_\-time\mbox{]}

return\-\_\-data = response\mbox{[}'qr\-\_\-centers'\mbox{]} \section*{Check if the returned data are equal to the expected}

if self.\-valid\-\_\-results == return\-\_\-data\-: return \mbox{[}True, self.\-elapsed\-\_\-time\mbox{]} else\-: return \mbox{[}\char`\"{}\-Unexpected result \-: \char`\"{} + str(return\-\_\-data), self.\-elapsed\-\_\-time\mbox{]} ```

Notice that the validate() function has a parameter named response. This is the response object returned from the Platform Service call. It is critical to return validation results as presented above!!

```python return \mbox{[}True, self.\-elapsed\-\_\-time\mbox{]} ```

If the response was succesfully validated,

```python return \mbox{[}\char`\"{}\-Unexpected result \-: \char`\"{} + str(return\-\_\-data), self.\-elapsed\-\_\-time\mbox{]} ```

if the validation failed.

The \href{https://docs.python.org/2/library/timeit.html}{\tt timeit} python module is used to time the execution of the test. {\bfseries This will be raplaced in v0.\-6.\-0 release and execution time of each test will be calculated by the rapp-\/testing-\/core-\/engine.}

Implement the execution function. The execution function is called by the test-\/core-\/engine to execute implemented test. The execution function must\-:


\begin{DoxyItemize}
\item Call the, {\bfseries to-\/test}, Platform Web Service through the Python R\-A\-P\-P-\/\-A\-P\-I call.
\item Invoke the validation function.
\end{DoxyItemize}

```python def execute(self)\-: start\-\_\-time = timeit.\-default\-\_\-timer() \section*{Call the Python Rapp\-Cloud service}

response = self.\-rapp\-Cloud.\-qr\-\_\-detection(self.\-file\-\_\-uri) end\-\_\-time = timeit.\-default\-\_\-timer() self.\-elapsed\-\_\-time = end\-\_\-time -\/ start\-\_\-time return self.\-validate(response) ```

You will notice that the above implementation of the execute() method calls the qr\-\_\-detection Service!!

```python response = self.\-rapp\-Cloud.\-qr\-\_\-detection(self.\-file\-\_\-uri) ``` 