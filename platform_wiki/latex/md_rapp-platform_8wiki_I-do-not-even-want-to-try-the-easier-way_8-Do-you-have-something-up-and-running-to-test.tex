If you do not want to setup / install the R\-A\-P\-P Platform, or even use the ready-\/to-\/deploy ova file we provide, but you want to use some of its functionalities, you can do so by invoking the R\-A\-P\-P Platform services in our already deployed instance.

The I\-P address of this instance is {\ttfamily 155.\-207.\-19.\-229} and you can invoke any H\-O\-P service using the following url\-:

\begin{quotation}
155.\-207.\-19.\-229\-:9001/hop/\-H\-O\-P\-\_\-\-S\-R\-V\-\_\-\-N\-A\-M\-E

\end{quotation}


You can find the H\-O\-P service names \href{https://github.com/rapp-project/rapp-platform/wiki/RAPP-HOP-Web-services}{\tt here}. Also since we use authentication you must set a \href{https://github.com/rapp-project/rapp-api/tree/master/python#authenticationtokens}{\tt R\-A\-P\-P Platform application token} if you want to utilize the R\-A\-P\-P Platform A\-P\-I.

Below are examples of calling deployed services, using {\ttfamily curl} cli (note that the token is passed as header parameter in the call -\/ in the A\-P\-I case this is done automatically)\-:

\paragraph*{ontology\-\_\-subclasses\-\_\-of}

```bash curl -\/\-X P\-O\-S\-T -\/d 'json=\{\char`\"{}ontology\-\_\-class\char`\"{}\-:\char`\"{}\-Oven\char`\"{}, \char`\"{}recursive\char`\"{}\-: true\}' -\/\-H \char`\"{}\-Accept-\/\-Token\-:rapp\-\_\-token\char`\"{} 155.\-207.\-19.\-229\-:9001/hop/ontology\-\_\-subclasses\-\_\-of ```

will respond with

```http \{\char`\"{}results\char`\"{}\-:\mbox{[}\char`\"{}http\-://knowrob.\-org/kb/knowrob.\-owl\#\-Oven\char`\"{},\char`\"{}http\-://knowrob.\-org/kb/knowrob.\-owl\#\-Microwave\-Oven\char`\"{},\char`\"{}http\-://knowrob.\-org/kb/knowrob.\-owl\#\-Regular\-Oven\char`\"{},\char`\"{}http\-://knowrob.\-org/kb/knowrob.\-owl\#\-Toaster\-Oven\char`\"{}\mbox{]},\char`\"{}error\char`\"{}\-:\char`\"{}\char`\"{}\} ```

\paragraph*{face\-\_\-detection}

```bash curl -\/v -\/\-X P\-O\-S\-T -\/\-F \char`\"{}json=\{\char`\"{}fast\char`\"{}=true\}\char`\"{} -\/\-F \char`\"{}file=@lenna.\-jpg\char`\"{} -\/\-H \char`\"{}\-Accept-\/\-Token\-:rapp\-\_\-token\char`\"{} 155.\-207.\-19.\-229\-:9001/hop/face\-\_\-detection ```

will respond with

```http \{\char`\"{}faces\char`\"{}\-:\mbox{[}\{\char`\"{}up\-\_\-left\-\_\-point\char`\"{}\-:\{\char`\"{}x\char`\"{}\-:95.\-0,\char`\"{}y\char`\"{}\-:89.\-0\},\char`\"{}down\-\_\-right\-\_\-point\char`\"{}\-:\{\char`\"{}x\char`\"{}\-:171.\-0,\char`\"{}y\char`\"{}\-:165.\-0\}\}\mbox{]},\char`\"{}error\char`\"{}\-:\char`\"{}\char`\"{}\} ```

You might have to read \href{https://github.com/rapp-project/rapp-platform/tree/master/rapp_web_services/services#service-specifications---request-arguments-and-response-objects}{\tt these} specifications on how to do valid P\-O\-S\-T requests to the Web Services 